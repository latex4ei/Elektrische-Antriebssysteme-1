\documentclass[fs, german]{latex4ei_fs}
\usepackage[european]{circuitikz}
\usepackage{pbox}
\begin{document}
\fstitle{Elektrische Antriebssysteme I} 

\section{Allgemeines}
\begin{sectionbox}
	\subsection{Einheiten}
	\begin{tabular}{llll} \ctrule
		\textbf{Größe} & \textbf{Definition} & \textbf{Einheit} & \textbf{SI-Notation} \\ \cmrule
		%Geschwindigkeit & $\vec{v}=\frac{\mathrm d \vec{s}}{\mathrm dt}$ & $\frac{m}{s}$ & \\
		%Beschleunigung & $\vec{a}=\frac{\mathrm d \vec{v}}{\mathrm dt}$ & $\frac{m}{s^2}$ & \\
		Frequenz & $f = \frac{c}{\lambda}$ & Hertz & $\si{\hertz} = \si{ \per \second}$\\
		Kraft & $ \vec F := m \cdot \vec a $ & Newton & $\si{\newton} = \si{\kilogram \meter \per \second \squared}$\\
		Druck & $p := \frac{\vec F_\perp}{A}$ & Pascal & $\si{\pascal} = \si{\newton \per \meter \squared} = \si{\kilogram \per \meter \second \squared}$\\
		${}^{\textstyle \text{Arbeit,}}_{\textstyle \text{Energie}}$ & $W := \int \vec F \diff \vec s$ & Joule & $\si{\joule} = \si{\newton\meter} = \si{\kilogram\meter \squared \per \second \squared}$ \\
		Leistung & $P := \frac{\mathrm dW}{\mathrm dt}$ & Watt & $\si{\watt} = \si{\joule \per \second} = \si{\kilogram\meter \squared \per \second \cubed}$\\
		%Impuls & $\vec p := m \cdot \vec v$ & $\frac{kg\; m}{s}$ & \\
		Spannung & $U := \frac{W}{Q}$ & Volt & $\si{\volt} = \si{\watt \per \ampere} = \si{\kilogram\meter  \squared \per \ampere\second \cubed}$\\
		Ladung & $Q:= \int I \diff t$ & Coulomb & $\si{\coulomb} = \si{\ampere\second}$\\
		Resistivität & $R := \frac{\diff U}{\diff I}$ & Ohm & $\si{\ohm} = \si{\volt \per \ampere} = \si{\kilogram\meter  \squared \per \ampere  \squared\second \cubed}$\\
		Kapazität & $C:= \frac{\diff Q}{\diff U}$ & Farad & $\si{\farad} = \si{\coulomb \per \volt} = \si{\ampere  \squared  \second \tothe{4}\per \kilogram\meter \squared}$\\	
		Induktivität & $L := \frac{\diff \Phi}{\diff I}$ & Henry & $\si{\henry} = \si{\volt\second \per\ampere} = \si{\kilogram\meter  \squared \per \ampere  \squared \second  \squared}$\\
		${}^{\text{Magnetischer}}_{\textstyle \text{Fluss}}$ & $\Phi_{\ir M} := \int \vec B \diff \vec A$ & Weber & $\si{\weber} = \si{\volt\second} = \si{\kilogram\meter \squared \per \ampere\second \squared}$\\[0.2em]
		${}^{\text{Magnetische}}_{\textstyle \text{Flussdichte}}$ & $\vec B := \mu_0 \vec H$ & Tesla & $\si{\tesla} = \si{\weber \per \meter  \squared} = \si{\kilogram \per \ampere\second \squared}$\\
\end{tabular}
\end{sectionbox}
\begin{sectionbox}
	\subsection{Newtonsche Mechanik $\vec F=m \vec a $}
	\begin{tabular}{ccc} \ctrule
				& \large Translation & {\large Rotation} (Radius $r$) \\ \cmrule
		Strecke/Winkel & \large $\vec x$ & \large $\vec \varphi =  \frac{\vec x}{r}$\\[0.2em]
		Geschwindigkeit & \large $\vec v = \dot{\vec x}$ & \large $\vec \omega = \dot{\vec \varphi} = \frac{\vec v}{r}$ \\[0.2em]
		Beschleunigung & \large $\vec a = \dot{\vec v} = \ddot{\vec x}$ & \large $\vec \alpha = \dot{\vec \omega} = \ddot{\vec \varphi} = \frac{\vec a}{r}$ \\[0.2em]
		Masse/Trägh. & \large $m$ & \large $\Theta = \int_V \vec r^2_\perp \diff m$ \\[0.2em]
		Impuls/Drall & \large $\vec p =m \vec v$ & \large $\vec L = \ma \Theta \vec \omega = \vec r \times \vec p$ \\[0.2em]
		Kraft/Moment & \large $\vec F = \dot{\vec p} = m \vec a$ &  \large $\vec M = \dot{\vec L} = \ma \Theta \vec \alpha = \vec r \times \vec F$ \\[0.2em]
		Energie & \large $E_{\ir kin}=\frac12mv^2$ & \large $E_{\ir rot}=\frac12 \Theta \omega ^2$\\[0.2em]
		Leistung & \large $P = \vec F \bdot \vec v$ & \large $P = \vec M \bdot \vec \omega$\\ \cbrule
	\end{tabular}
\end{sectionbox}
\begin{sectionbox}
	\subsection{Elektrische Felder}

	\subsubsection*{Maxwellsche Gleichungen (Naturgesetze)}
	\begin{tabular}{ll}
		Gaußsches Gesetz: & Faradaysches ind. Gesetz\\
		\large $\div \vec D = \varrho $ & \large $\rot \vec E + \frac{\partial \vec B}{\partial t} = 0$ \\[1em]
		Quellfreiheit des magn. Feldes & Ampèrsches Gesetz\\
		\large $\div \vec B = 0$ & \large $\rot \vec H = \vec j + \frac{\partial \vec D}{\partial t}$\\[0.3em]
	\end{tabular} 

	\begin{tabular}{lll} \ctrule
		\textbf{Resistiv} & \textbf{Kapazitiv} & \textbf{Induktiv}\\ \cmrule
		\large $\mathrm d I = G \diff U$ & \large $\mathrm d Q = C \diff U$ & \large $\mathrm d \Phi_M = L \diff I$\\[0.3em] 
		\large $\vec j = \sigma \vec E$ & \large $\vec D = \varepsilon \vec E$ & \large $\vec B = \mu \vec H$\\ [0.3em] 
		\large $\mathrm d I = \vec j \diff A$ & \large $\mathrm d U = \vec E \diff \vec r$ & \large $\mathrm d \Phi_M = \vec B \diff A$\\[0.3em]  
		\large $\vec j = q n \vec v$ & \large $Q(V) \equiv \oiint\limits_{\partial V}\! \vec D \diff \vec A\;$ & \large $I(A) \equiv \oint\limits_{\partial A}\! \vec H \diff\vec r$\\ \noalign{\vspace{2pt}}\cmrule
		Widerst. $R = \rho \frac{l}{A}$ & Kondensator $C=\varepsilon \frac{A}{d}$ & Spule $L=\mu A \frac{N^2}{l}$\\
		\cbrule
	\end{tabular}

\end{sectionbox}	
\section{Gleichstrommaschine}

\begin{sectionbox}

\subsection*{Elektrisch}

Ankerkreis: $U_a = U_i + R_a I_a + L_a \frac \diff I_a \diff t$ \\
Erregerkreis: $U_e = R_e I_e + L_e \frac{\diff I_e}{\diff t}$

\subsection*{Mechanisch}
$M_{\ir el} = M_{\ir Welle} + M_R + J \frac{\diff \omega_m}{\diff t}$

\subsection*{Kopplung}
$U_i = c \Phi \omega_m$ \\
$M_{\ir el} = c \Phi I_a$ \quad mit $c$: Maschinenkonstante

Erregerfluss: $\Phi = \frac{L_e}{N_e} I_e$ \quad mit $N_e$: Windungszahl


\subsection*{Wirkungsgrad}
$\eta = \frac{P_m}{P_e} = \frac{M \omega_m}{U_a I_a + U_e I_e}$
\end{sectionbox}

\begin{sectionbox}
\subsection{Fremderregte Maschine}

\subsubsection*{Stationärer Betrieb}
$\omega_m =  \frac{U_a - R_a I_a}{c \Phi} $ \\
$M_{\ir el} = c \cdot \Phi \cdot I_a$ \\

$\omega_m = \frac{U_a}{c \Phi} - \frac{R_a}{(c \Phi)^2} M_{\ir el}  $
\subsubsection*{Leerlauf}
 $M = 0 \Ra I_a = 0 \Ra U_a = U_i$ \\
 mechanische Kreisfrequenz: $\omega_{m0} = \frac{U_a}{c \Phi}$

 Leerlaufdrehzahl: $n_0 = \omega_{m0} \cdot \frac{60}{2 \pi}$ (U/min)
\end{sectionbox}
\begin{sectionbox}
\subsection{Nebenschlussmaschine}

$\Phi = \frac{L_e}{N_e} \cdot I_e = \frac{L_e}{N_e} \cdot \frac{U_a}{R_e + R_V}$

Leerlaufdrehzahl: $w_{m0} = \frac{U_a}{c \Phi} = \frac{N_e (R_e + R_V)}{c \cdot L_e}$

Drehzahlabhänigkeit:\\ $w_m = \frac{N_e (R_e + R_V)}{c \cdot L_e} - \frac{R_a (R_e + R_V)^2 \cdot N_e^2}{(c \cdot L_e \cdot U_a)^2} \cdot M_{\ir el}$
\end{sectionbox}
\begin{sectionbox}
\subsection{Seriemaschine}

$I_e$ = $I_a$ = $I$

$\Phi = \frac{L_e}{N_e} \cdot I_e = \frac{L_e}{N_e} \cdot I_a$

$\omega_m = \frac{U- (R_a + R_e) I_a}{c \frac{L_e}{N_e} I_a}$

Drehmoment: $M = c \frac{L_e}{N_e} I_a^2 $

Ohne Last: $I_a \ra 0 \Ra \omega_m \ra \infty$
\end{sectionbox}

\section{Universalmotor}
Gleichstrommaschine: Aber geblecht (wegen magnetischen Streufeldern)


Erregerspannung: $\vec U_e = \vec I \cdot R_e + \j \omega L_e \vec I$

Ankerspannung: $\vec U_a = \vec I \cdot R_a + \j \omega L_a \vec I$

$\vec U_i = c \vec \Phi_{\ir de} \omega_m$ \\


$\vec U = \vec U_i + \vec I (R_a + R_e) + \j \omega \vec I (L_a + L_e)$

$M_{\ir el} = c \Phi_{\ir de} I = c \frac{L_e}{N_e} I^2$

Drehzahlabhängigkeit:

$\omega_m = \frac{U- (R_a + R_e) \vec I - \j \omega (L_a + L_e) \vec I}{\sqrt{c \frac{L_e}{N_e}M_{\ir el}}}$ \\
$\omega_m \propto \frac{U_i}{\sqrt{M_{\ir el}}}$

Typische Auslegungen der Laufzeiten: Bohrmaschine ($50 \si{\hour}$)

\section{Drehfeldmaschinen}


\section{Synchronmaschine}

\begin{sectionbox}

Polpaarzahl: $\omega_{\ir mech} = \frac{\omega_1}{p}$ \quad $M = p \cdot M_{p}$ 

mit $M_p$: Drehmoment pro Polpaar

Polradspannung $\vec U_p = \j \omega_1 \vec \Psi_p$: \\

 $\abs{\vec U_p} = \omega_1 \abs{\vec \Psi_p}$\\


 $\vec U_1 = \vec U_p + \Delta \vec U$
 $\Delta \vec U = \j X_d \cdot \vec I_1 (+ R \cdot \vec I_1)$ \\



Leistung:

 $P = \Re{\vec S}  = \Re{ 3 \vec U \vec I^* } = \Re{3( U_d + \j U_q ) (I_d - \j I_q)}$ \\ $= 3 (U_d I_d + U_q I_q)$ \\

 d := direct (in Flussrichtung)

 $U_{1q} = \abs{\vec U_1} \cos \theta$

 $U_{1d} = - \abs{\vec U_1} \sin \theta$
\subsection{Schenkelpolmaschine}

Statorspannung:

$\vec U_1 = \vec I_1 R_1 + U_{xd} + \j U_{xq} + \j U_P = \vec I_1 R_1 - \omega_1 L_q I_q + \j \omega_1 L_d I_d + \j U_P$

Stromkomponenten:

$I_{1d} = \frac{U_1 X_q \cos \theta - U_1 R_1 \sin \theta - U_P X_q}{R_1^2 + X_d X_q}$

$I_{1q} = \frac{U_1 \sin \theta + R_1 I_{1d}}{X_q} = \frac{U_1 R_1 \cos \theta + U_1 X_d \sin \theta - U_P R_1}{R_1^2 + X_d X_q}$ 

Leistung:
 $P_m = M \omega_{\ir mech}$

Moment: 

$M = \frac{3 U_1 U_P}{\omega_{\ir mech} X_d} \sin (\theta ) + \frac{3 U_1^2}{2 \omega_{\ir mech}} (\frac{1}{X_q} - \frac{1}{X_d} ) \sin ( 2 \theta )$

 \subsection{Vollpolmaschine}

Statorspannung:
$\vec U_1 = \vec I_1 (R_1 + \j \omega_1 L_d) + \j U_P$

Leistung:
 $P_m = M \omega_{\ir mech}$

Moment: 
$M = \frac{3 U_1 U_P}{\omega_{\ir mech} X_d} \sin ( \theta ) $
\end{sectionbox}
\begin{sectionbox}
\subsection{Raumzeigerdarstellung}
Statorspannung:\\
$U_d = R_1 I_d + \frac{\diff \Psi_d}{\diff t} - p \omega_{\ir mech} \Psi_q$\\
$U_q = R_1 I_q + \frac{\diff \Psi_q}{\diff t} + p \omega_{\ir mech} \Psi_d$

Statorfluss: \\
$\Psi_d = L_d I_d + L_{dE} I_E + L_{dD} I_D$ \\
$\Psi_q = L_q I_q +  L_{qQ} I_Q$


Dämpferspannung:

$0 = R_D I_D + \frac{\diff \Psi_D}{\diff t}$ \\
$0 = R_Q I_Q + \frac{\diff \Psi_Q}{\diff t}$

Dämpferfluss:

$\Psi_D = L_{dD} I_d + L_{ED} I_E + L_D I_D$ \\
$\Psi_Q = L_{qQ} I_q + L_{Q} I_{Q}$

Erregerspannung:
$U_E = R_E I_E + \frac{\diff \Psi_E}{\diff t}$

Erregerfluss:
$\Psi_E = L_{dE} I_d + L_E I_E + L_{ED} I_D$

Drehmoment:

$M = \frac{3 p}{2} (I_q \Psi_d - I_d \Psi_q)$

Drehzahl:

$\omega_{\ir mech} = \frac{1}{J} \int (M- M_{\ir Last}) \diff t$

 
 \end{sectionbox} 

 \section{Asynchronmaschine}

 \begin{sectionbox}
\ctikzset{bipoles/length=3em}
 \begin{circuitikz}[scale = 0.8] \draw
 (0,0) to[short, o-, i=$I_1$] (1,0) to[R, l=$R_1$] (2,0) to [L, l=$L_{\sigma 1}$] (3,0) -- (3,0)
 (3,0) to[L, l=$L'_{\sigma2}$] (4,0) to[R, l=$R'_2$] (5,0) to[R, l=$R'_S$] (6,0) to[short, i<=$I'_2$, , -o] (7,0 )
 (3,0) -- (3, -0.6) -- (2.5, -0.6) to[L, l=$L_h$] (2.5, -2.3) --  (3, -2.3)
 (3, -0.6) -- (3.5, -0.6) to[R, l=$R_{\ir Fe}$] (3.5, -2.3) --  (3, -2.3) -- (3, -3)
 (0,-3) to[short, o-o] (7,-3)
 (0,0) to[open, v=$U_1$] (0,-3)
 (7,0) to[open, v^=$U'_2 / s$] (7,-3)
  (2,0) to[open, v=$U_{h1} $] (2,-3)
   (4.3,0) to[open, v^=$U'_{h2}$] (4.3,-3);
 \end{circuitikz}
 Drehfeld der Satorwicklung: $\omega_{D1} = \frac{\omega_{1}}{p}$

 Drehfeld des Rotors: $\omega_{D2} = \omega_{D1} - \omega_{\ir mech}$

$\omega_2 = p \cdot \omega_{D2}$

Schlupf: $s = \frac{n_{\ir syn} - n}{n_{\ir syn}} = \frac{\omega_1 - p \cdot \omega_{\ir mech}}{\omega_1} = \frac{\omega_2}{\omega_1}$ 

 \subsection{Leerlauf}
  $\omega_{D2} = 0 \ra \omega_{\ir mech} = \omega_{\ir syn} = \omega_{D1} = \omega_1 /p $ 

  Im Leerlauf ist der Schlupf 0 $s = 0$

  \subsection{Rotierendes Koordinatensystem}

Statorspannungsgleichung:

  $\vec u_1 = R_1 \vec i_1 + \frac{\diff \vec \Psi_1}{\diff t} + \j \omega_K \vec \Psi_1$

Rotorspannungsgleichung:

  $\vec u_2 = R_2 \vec i_2 + \frac{\diff \vec \Psi_2}{\diff t} + \j (\omega_K - p \omega_{\ir mech} ) \vec \Psi_2$


Statorfluss:

  $\Psi_1 = L_1 \vec i_1 + L_h \vec i_2$

Rotorfluss:

  $\Psi_2 = L_2 \vec i_2 + L_h \vec i_1$

Drehmoment:

  $M = \frac{3 p}{2} \Im{\vec \Psi_1^* \vec i_1} = \frac{3p}{2} \frac{L_h}{L_2} \Im{\vec \Psi_1^* \vec i_1}$ \\$ =  \frac{3p}{2} \frac{L_h}{L_1} \Im{\vec \Psi_1^* \vec i_2} $

  $\omega_{\ir mech} = \frac{1}{J} \int (M - M_{\ir Last}) \diff t$
 \end{sectionbox}

 \section{Umrichter}

 \begin{sectionbox}
 \subsection{B2}
$U_{\ir di0} = \frac{1}{T} \int \limits_0^T u_d \cdot \diff t = \frac{1}{\pi} \int \limits_0^\pi u_d \cdot \diff (\omega t) = \frac{2 \sqrt 2}{\pi} U_N = 0.900 U_N$
 
\subsubsection{Gesteuert}
Mittelwert der Ausgangsspannung in Abhängigkeit des Zündwinkels $\alpha$:

$U_{\ir di \alpha} = \frac{1}{\pi} \int \limits_\alpha^{\alpha + \pi} u_d \cdot \diff (\omega t) = U_{di 0} \cdot \cos \alpha$ 
 \subsection{B6}
 Idealisierte DC-Spannung (Mittelwert): 

$U_{\ir di0} = \frac{3 \sqrt 2}{\pi} \cdot U_N = 1.35 U_N$

Effektivwert des Netzstromes:

$i_{N, \ir eff} = \sqrt{\frac{2}{3}} I_d $

Halbleiterstrom:

Effektivwert: $i_{\ir HL, eff} = \sqrt{\frac{1}{3}} I_d$

Mittelwert: $i_{\ir HL,  avg} = \frac{1}{3} I_d$

\subsubsection{Gesteuert}

$U_{\ir di \alpha} = U_{\ir di0} \cdot \cos \alpha$

Leistungsfaktor: $\lambda = 0.955 \cdot \cos \alpha$
 \end{sectionbox}

 \section{Halbleiter}
\begin{sectionbox}

 \begin{tablebox}{lll}
 & $25 \si{\celsius} (\text{ca. } 300 \si{\kelvin})$ & $125 \si{\celsius} (
 \text{ca.  } 400 \si{\kelvin})$ \\ 
\cmrule
 $n_{io}$ & $1.5 \cdot 10^{10} \si{1\per \centi \meter^3}$ & $10^{13} \si{1 \per \centi \meter^3}$ \\
 $\rho$ & $2.5 \cdot 10^5 \si{\ohm \centi \meter}$ &$3.5 \cdot 10^2 \si{\ohm \centi \meter}$
 \end{tablebox}
\begin{symbolbox}
\begin{tabular}{cc}
$n_{io}$ & Dichte der freien Elektronen im Leitungsband \\
$p_{io}$ & Dichte der freien Löcher im Valenzband \\
$k$ & Boltzmann Konstante \\
$E_{\ir Fi}$ & Fermi-Niveau (Energieniveau mit $W_{\ir FD} (E_{\ir Fi})= 0.5$)
\end{tabular}
\end{symbolbox}

Fermi-Dirac Funktion:

$W_{\ir FD} (E) = \frac{1}{1 + \exp\left(\frac{E-E_{\ir Fi}}{k \cdot T}\right)}$

gibt die Besetzungswahrscheinlichkeit eines Energiezustandes im idealen Elektronengas an.

Bewegungsgleichung für Elektronen:

$m \cdot \left(\frac{\diff v}{\diff t} + \omega_0 \cdot v \right) = F_A = - e_0 \cdot E$
\begin{symbolbox}
\begin{tabular}{cc}
$m$ & effektive Masse \\
$\omega_0$ & Reibungskoeffizient \\
$v$ & Driftgeschwindigkeit \\
$F_A$ & Äussere Kraft \\
$e_0$ & Ladung \\
$E$ & Elektrische Feldstärke
\end{tabular}
\end{symbolbox}

Stationäre Geschwindigkeit der Ladungsträger:
$v = - \frac{e_0}{m \cdot \omega_0} \cdot E = \mu \cdot E$

Elektrische Leitfähigkeit:
$\sigma = e_0 (n \mu_n + p \mu_p)$

\begin{symbolbox}
\begin{tabular}{cc}
$n$ & Konzentration der freien Elektronen \\
$p$ & Konzentration der freien Löcher \\
$\mu_n$ & Beweglichkeit der freien Elektronen \\
$\mu_p$ & Beweglichkeit der freien Löcher \\
$n_i$ & Intrinsische Ladungsträgerdichte
 \end{tabular}
\end{symbolbox}

Ladungsträgerdichte im intrinsischen Halbleiter:
 $n_i^2 = n \cdot p$
\end{sectionbox}
\begin{sectionbox}
 \subsection{n-Dotierung}
 Dotierung mit Donatoren (Elektronen) $\Ra$ Elektronenleitung \\

Anhebung des Ferminiveaus $\Ra$ mehr Ladungsträger im Leitungsband

Dichte der freien Elektronen: $n_{n0} = p_{n0} + N_D^+$ \\
mit $p_{n0}$ verursacht durch Eigenleitung \\

Dichte der freien Löcher: $p_{n0} \approx \frac{n_i^2}{N_D^+}$

Elektrische Leitfähigkeit: $\sigma = e_0 \cdot N_D^+ \cdot \mu_n$
\end{sectionbox}
\begin{sectionbox}

 \subsection{p-Dotierung}
Dotierung mit Akzeptoren (Löchern) $\Ra$ Löcherleitung \\

Absenkung des Ferminiveaus $\Ra$ mehr Ladungsträger im Valenzband

Dichte der freien Elektronen: $p_{p0} = n_{p0} + N_A^-$ \\
mit $n_{p0}$ verursacht durch Eigenleitung \\

Dichte der freien Löcher: $n_{p0} \approx \frac{n_i^2}{N_A^-}$

Elektrische Leitfähigkeit: $\sigma = e_0 \cdot N_A^- \cdot \mu_p$

\end{sectionbox}
\begin{sectionbox}

\subsection{pn-Übergang}

Diffusionsgleichungen: \\
$i_{n, \ir Diff} = \epsilon_0 \cdot D_n \cdot \frac{\diff n}{\diff x}$\\
$i_{p, \ir Diff} = \epsilon_0 \cdot D_p \cdot \frac{\diff p}{\diff x}$

Diffusionskonstanten: 
$D_n = \mu_n \frac{k \cdot T}{\epsilon_0}$ \quad $D_p = \mu_p \frac{k \cdot T}{\epsilon_0}$

Poissongleichung: \\
 $\frac{\diff^2 V}{\diff x^2} = - \frac{\diff E}{\diff x} = - \frac{\rho}{\epsilon} = \frac{e_0}{\epsilon} \cdot \left[ n(x) - p(x)  N_D^+ (x) + N_A^- (x) \right]$

 Stromdichte-Gleichgewicht: \\
 $i_p (x) = e_0 \cdot \mu_p \cdot p \cdot E - \epsilon_0 \cdot D_p \cdot \frac{\diff p}{\diff x}$ \\
  $i_n (x) = e_0 \cdot \mu_n \cdot n \cdot E - \epsilon_0 \cdot D_n \cdot \frac{\diff n}{\diff x}$
\end{sectionbox}

\section{Leistungshalbleiter}

\begin{sectionbox}
\subsection{Diode}
\end{sectionbox}

\begin{sectionbox}
\subsection{Thyristor}
\includegraphics[width=\columnwidth]{Thyristor.pdf}

\end{sectionbox}

\begin{sectionbox}
\subsection{IGCT (Integrated Gate Commutated Thyristor)}
\subsubsection*{Aufbau}
\includegraphics[width=\columnwidth]{IGCT.pdf}

\subsubsection*{Ansteuerung}
\includegraphics[width=0.7\columnwidth]{IGCT_Ansteuer.pdf}

\subsubsection*{Ersatzmodell}
\includegraphics[width=0.7\columnwidth]{IGCT-Ersatz.pdf}

\includegraphics[width=0.7\columnwidth]{IGCT_Turnoff.pdf}

\includegraphics[width=0.7\columnwidth]{IGCT-TO-Snubber.pdf}



\end{sectionbox}

\begin{sectionbox}
\subsection{IGBT}
\subsubsection*{Aufbau}
\includegraphics[width=\columnwidth]{IGBT.pdf}

\subsubsection*{Ansteuerung}
\includegraphics[width=\columnwidth]{IGBT_Ansteuer.pdf}

\end{sectionbox}

\section{Pulswechselumrichter}
\begin{sectionbox}

\includegraphics[width=\columnwidth]{U-Umrichter.pdf}
(Amplituden-) Modulationsindex (Aussteuerung): $m_a = \frac{\hat U_{mA}}{\hat U_{CR}}$

Frequenzmodulationsverhältnis: $m_f = \frac{f_{CR}}{f_1}$

\subsection{Grundfrequenztaktung}
Common Mode Spannung: \\
$u_{CM} (t) = u_{M0} (t) = \frac{1}{3} (u_{R0} + u_{S0} + u_{T0})$

Spitzenwert der Phasenspannung:
$\hat U_{R0_1} = \frac{2}{\pi} U_d$


Amplitude der Grundwelle der verketteten Ausgangsspannung: \\
$U_{LL_1} = \frac{\sqrt 6}{\pi} U_d$

Oberschwingungen der Ausgangsspannung: \\
$U_{LL_v} = \frac{\sqrt 6}{\pi} \frac{U_d}{v}$ für $v = 6n \pm 1 (n = 1, 2, 3, \ldots)$
\subsection{PWM}  
\end{sectionbox}

\section{U/f Steuerung}

\section{Auslegung von Regelkreisen}
\begin{sectionbox}
Offener Regelkreis: $\vec G_0 (s) = \frac{y(s)}{u(s)} = \vec G_{S1} (s) \cdot \vec G_{S2} (s)$

Offener korrigierter Regelkreis: $\vec G_K (s) = \frac{y(s)}{e(s)} = \vec G_R (s) \cdot \vec G_0 (s)$

\subsection{Stabilität}

\subsubsection{Nyquist Ortskurve}
	
	die Nyquist Ortskurve ist die Frequenzgangortskurve des offenen Regelkreises $G_o (s)$ 
	
	\subsubsection{Nyquist Kriterium}
	
	Betrachtet Nyquist-Ortskurve und Pole $p_{\ir links}, p_{\ir rechts}, p_{\ir auf}$ im Bezug auf die Imaginärachse.
	Das System ist stabil falls die OK nicht durch $-1 + \j 0$ verläuft und die Phasenänderung \\
	
	\begin{emphbox}
	$W_{\ir ist} = \mathop{\Delta}\limits_{\omega = 0}^{\omega \ra \infty} \Phi$
	\end{emphbox}
	
	von $-1 + \j 0$ aus gesehen gleich \\
	
		\begin{emphbox}
		$W_{\ir soll} = \pi p_{\ir rechts} + \frac{\pi}{2} p_{\ir auf}$
		\end{emphbox}
	ist; also $W_{\ir soll} \stackrel{!}{=} W_{\ir ist}$ 
\end{sectionbox}

\begin{sectionbox}

\subsection{Stationäre Genauigkeit} 

Im stationären Betrieb soll der Regelfehler sehr klein werden:
$e (t \ra \infty) = 0$

Im Frequenzbereich:
$e (t \ra \infty) = \lim \limits_{s \ra 0} s \cdot e(s) = \frac{1}{1 + \vec G_K (0)}$
\end{sectionbox}


\subsection{Frequenzgangfunktion  $G(\j \omega)$}
Beschreibt die Auswirkungen von sinusförmigen Anregungen auf die Systemantwort.

Die Auswirkungen auf Amplitude $A$ und Phasenverschiebung $\varphi$ ergeben die Frequenzgangfunktion $G(\j \omega)$

	\begin{emphbox}
	$G(\j \omega) = A(\omega) e^{\j \varphi (\omega)} = \Re {G(\j \omega)} + \j \Im{G(\j \omega)}$ 
	\end{emphbox}
	\pbox{4cm}{ \includegraphics{./img/OK_GO.pdf} } \quad
	\pbox{4cm}{
		\emph{Einschwingzeit:} $T_{\ir Ein} \approx \frac{3}{\omega_{\ir D1}}$\\
		\\
		Stabilitätskriterien mit Totzeit:\\
		geschl. RK ist E/A \textbf{stabil} falls:\\
		Phasenrad $\Phi_{\ir R} > 0$\\
		Amplitudenrand $A_{\ir R} > 1$\\
		\\
		$\Psi_{\ir R} \approx 30^\circ \Leftrightarrow $ gutes Störverhalten\\
		$\Psi_{\ir R} \approx 60^\circ \Leftrightarrow $ gutes Folgeverhalten
	}

	\begin{tabular}{@{}ll@{}}
	\textbf{Kenngrößen:} & Bodediagramm:\\ \mrule
		Amplituden-Durchtrittsfrequenz $\omega_{D1}$: & $A(\omega_{D1}) = 1$\\
		Phasen-Durchtrittsfrequenz $\omega_{D2}$ & $\varphi (\omega_{D2}) = - \pi = -180^\circ$\\
		Phasenrand/Phasenreserve $\Psi_R$: & $\Psi_R = \varphi (\omega_{D1}) + \pi$\\
		Amplitudenrand/-reserve $A_R$ $(= K_{\ir krit})$ & $\frac{1}{A_R} = A (\omega_{D2})$
	\end{tabular}

\subsection{P-Regler}
$\vec G_R (s) = \vec G_P (s) = K_R$

Design: Ändere die Durchtrittsfrequenz $\omega_D$ durch ändern von $K_R$ so, dass der gewünschte Phasenrand entsteht. \\

Für eine Strecke mit Tiefpasscharakteristik und 2 Zeitkonstanten:

$\varphi_{VZ1} = - \arctan (\omega T_{S2})$ \\
$\omega_D = \frac{1}{T_{S2}} \cdot \tan (\frac{\pi}{2} - \varphi_R)$

$\Ra K_R = \frac{\omega_D T_{S1}}{K_S}$

\subsection{PI-Regler}

$\vec G_R (s) = \vec G_{PI} (s) = K_R \cdot \frac{1 + s T_R}{s T_R}$  

Phasendrehung durch Integrator wird für hohe Frequenzen kompensiert.

Knickfrequenz $T_R$ und Verstärkung $K_R$ können unabhänig voneinander eingestellt werden. \\

Für eine Strecke mit Tiefpasscharakteristik und 2 Zeitkonstanten:

Setze die Zeitkonstante $T_R$ auf die grösste Zeitkonstante der Strecke:
$T_R = T_{S1}$. Stelle die Phasenreserve über $K_R$ ein (vgl. P-Regler).

\subsection{PID-Regler}

$\vec G_{PID} (s) = K_R \cdot \frac{1 + s T_{R1}}{s T_{R1}} \cdot \frac{1 + s T_{RD1}}{1 + s T_{RD2}}$ mit $T_{RD1} > T_{RD2}$ 

Phasenanhebende Funktion kann die Phasenreserve erhöhen.

Verhältnis der Knickfrequenzen: $k_{12} = \frac{T_{RD1}}{T_{RD2}}$

Maximaler Phasenwinkel: $\varphi_{\ir max} = \arctan (\sqrt{ k_{12}}) - \arctan (\frac{1}{\sqrt{k_{12}}})$

\section{Modellierung der Asynchronmaschine}

\section{Rotorflussorientierte Regelung} 

\section{Direct Torque Control} 

\section{Flussschätzung} 

\end{document}